\documentclass[]{article}
\usepackage{graphicx}
\graphicspath{{imagenes/}}
\usepackage[spanish]{babel}
\usepackage[a4paper, top=2.5cm, bottom=2.5cm, left=3cm, right=3cm]{geometry}
\usepackage[hidelinks]{hyperref}
\usepackage[T1]{fontenc}
\usepackage{listings}
\usepackage{xcolor}
\usepackage{float}
\usepackage{booktabs}
\usepackage{multirow}
\usepackage{amsmath}
\usepackage{longtable}
\usepackage{tabularx}

% Configurar el color de los enlaces
\hypersetup{
    colorlinks=true, % Activa el color en los enlaces
    linkcolor=blue,  % Color para los enlaces internos (por ejemplo, tablas de contenido)
    citecolor=black,  % Color para las citas bibliográficas
    filecolor=blue,  % Color para enlaces a archivos
    urlcolor=blue    % Color para los enlaces a URLs
}


\definecolor{miverde}{rgb}{0,0.6,0}
\definecolor{miazul}{rgb}{0.5,0.5,1}
\lstdefinelanguage{cql}{
  morekeywords={SELECT, FROM, WHERE, INSERT, INTO, UPDATE, DELETE, CREATE, TABLE, PRIMARY, KEY, IF, EXISTS, NOT, NULL, AND, OR, SET, USE, VALUES, IN, ALLOW, FILTERING, AS}, % Palabras clave
    ndkeywords={COUNT, AVG},
    ndkeywordstyle=\color{miazul},
  sensitive=false, % Si las palabras clave distinguen mayúsculas de minúsculas
  morecomment=[l]--, % Comentarios de línea (prefijados con "--")
  morecomment=[s]{/*}{*/}, % Comentarios de bloque
  morestring=[b]', % Cadenas entre comillas simples
  morestring=[b]" % Cadenas entre comillas dobles
}


\lstdefinestyle{cql}{
    language=cql,
    backgroundcolor=\color{gray!2},     % Color de fondo
    basicstyle=\ttfamily,               % Tipo y tamaño de fuente
    keywordstyle=\color{blue}\bfseries, % Color para palabras clave
    stringstyle=\color{miverde},        % Color para cadenas
    commentstyle=\color{red},           % Color para comentarios
    showspaces=false,                   % No mostrar espacios
    showstringspaces=false,             % No mostrar espacios en las cadenas
    frame=single,                       % Poner un marco alrededor del código
    breaklines=true,                    % Romper las líneas largas
    captionpos=b,                       % Posición del caption
    tabsize=4,                          % Tamaño de las tabulaciones
    escapeinside={\%*}{*)},             % Para incluir código LaTeX en los listings
    morekeywords={self},                 % Palabras clave adicionales
    extendedchars=true,
    inputencoding=utf8
}

\lstdefinestyle{python}{
    language=Python,
    backgroundcolor=\color{gray!2},     % Color de fondo
    basicstyle=\ttfamily,               % Tipo y tamaño de fuente
    keywordstyle=\color{blue}\bfseries, % Color para palabras clave
    stringstyle=\color{miverde},        % Color para cadenas
    commentstyle=\color{red},           % Color para comentarios
    showspaces=false,                   % No mostrar espacios
    showstringspaces=false,             % No mostrar espacios en las cadenas
    frame=single,                       % Poner un marco alrededor del código
    breaklines=true,                    % Romper las líneas largas
    captionpos=b,                       % Posición del caption
    tabsize=4,                          % Tamaño de las tabulaciones
    escapeinside={\%*}{*)},             % Para incluir código LaTeX en los listings
    morekeywords={self}                 % Palabras clave adicionales
}
%title
\title{Práctica 1} 

\author{Adrián Ferández Galán, César López Mantecón y Manuel Gómez-Plana Rodríguez}

\begin{document}

\begin{titlepage}
    \centering
   \includegraphics[width=0.9\textwidth]{uc3m.jpg} 
    {\Huge Universidad Carlos III\\
    
     \Large Arquitectura de Datos\\
     \vspace{0.5cm}
     Curso 2024-25}
    \vspace{2cm}

    {\Huge \textbf{Práctica 2} \par}
    \vspace{0.5cm}
    {\Large Resultados del desarrollo de la migración de la base de datos a \texttt{Cassandra} \par}
    \vspace{8cm}

   \textbf{Ingeniería Informática, Cuarto curso}\\
    \vspace{0.2cm} 
    Adrián Fernández Galán       (NIA: 100472182, e-mail: 100472182@alumnos.uc3m.es)\\
    César López Mantecón         (NIA: 100472092, e-mail: 100472092@alumnos.uc3m.es)\\
    Manuel Gómez-Plana Rodríguez (NIA: 100472310, e-mail: 100472310@alumnos.uc3m.es)
    \vspace{0.5cm}

   
    \textbf{Prof .} Lourdes Moreno López\\
    
    \textbf{Grupo: } 81   
    
\end{titlepage}
\newpage

\renewcommand{\contentsname}{\centering Índice}

\hypersetup{linkcolor=black}
\tableofcontents
\hypersetup{linkcolor=blue}
\newpage

\section{Introducción}
\label{sec:introduccion}
En este documento se recoge los resultados obtenidos durante el desarrollo de la práctica 2 de la asignatura
\textit{Arquitectura de Datos}. En esta práctica se tratará de completar una
migración de una base de datos desde \texttt{MongoDB} a \texttt{Cassandra}.
Además, se computarán nuevas tablas con el fin de permitir el análisis
estadístico, aprovechando las cualidades de \texttt{Cassandra} para el análisis
de datos gracias a su capacidad para la consulta masiva de datos de una misma
columna.

En el diseño de consultas se tratará de aprovechar al máximo las capacidades de
\texttt{Cassandra} en la lectura y escritura, dejando a la aplicación otra clase de
operaciones. De esta forma, ambos sistemas trabajarán en conjunto, garantizando
la eficiencia de la aplicación.


% =======
% Para la realización de esta práctica se busca realizar una migración de datos desde el gestor de bases de datos \textit{Mongodb} a \textit{Cassandra}, además de suplir una serie de casos de uso 
% 
% Esta práctica se basa en la migración de un sistema gestor de expedientes de sanciones que se encontraba en el gestor de bases de datos \textit{Mongodb} y se quiere mover a \textit{Cassandra}, además de implementar nuevos casos de uso. 
% Para ello se realizará un estudio de los datos almacenados en \textit{mongodb} y de los casos de uso propuestos. Tras esto se desarrollará un diseño lógico y físico de la base de datos en \textit{Cassandra} que nos faciliten la creación de las tablas en este mismo gestor de bases de datos y la implementación de las consultas enfocadas a los casos de uso.
% Con las tablas configuradas se insertarán los antiguos datos a través de la herramienta \textit{PySpark}, para finalmente probar el correcto funcionamiento del nuevo sistema.

\section{Prueba de las querys}
Para probar que las tablas de cassandra funcionan, es necesario comprobar que las querys explicadas con anterioridad funcionan de acuerdo a lo exigido por los casos de uso en el enunciado. Así, se ejecutarán las querys y se mostraran algunas filas que sirvan como ejemplo de salida.

\subsection{Query de las funciones operativas}
Para los resultados de las consultas de las funciones operativas se han eliminado algunas características con el objetivo de mostrar solo la información más relevante.
Se han obtenido los siguientes resultados para la query de las función operativa sobre las sancione generadas para este DNI = 63050842E:
\begin{table}[H]
\begin{longtable}{lllrllll}
    
    \hline
    \textbf{dni\_deudor} & \textbf{tipo} & \textbf{fecha\_grabacion} & \textbf{cantidad} &\textbf{estado} & \textbf{matricula} \\ 
    \endfirsthead

    \hline
    \textbf{dni\_deudor} & \textbf{tipo} & \textbf{fecha\_grabacion} & \textbf{cantidad} &\textbf{estado} & \textbf{matricula} \\ 
    \endhead
    
    63050842E & discrepancia carne & 2013-05-09 18:53:14 & 1000 & stand by & 2955AUA \\ \hline
    63050842E & discrepancia carne & 2013-05-10 18:53:14 & 1000 & stand by & 2955AUA \\ \hline
    63050842E & discrepancia carne & 2013-05-29 10:35:28 & 1000 & stand by & 2955AUA \\ \hline
    63050842E & impago             & 2008-03-12 00:46:59 &  140 & stand by & 2955AUA \\ \hline
    63050842E & impago             & 2008-03-12 02:04:01 &  190 & stand by & 2955AUA \\ \hline
    63050842E & impago             & 2008-09-09 14:26:53 &  370 & stand by & 2955AUA \\ \hline
    
    \end{longtable}
    \caption{Sanciones para el DNI = 63050842E}
\end{table}

Como se puede observar, se muestra la información más relevante relacionada con las sanciones del DNI: 63050842E \\

Se han obtenido los siguientes resultados para la query de las función operativa sobre los expedientes activos sin plazo de pago cerrado:

\begin{table}[H]
\begin{longtable}{lllll}
    \hline
    \textbf{dni\_deudor} & \textbf{tipo} & \textbf{fecha\_grabacion} & \textbf{estado} & \textbf{matricula} \\ \hline
    \endfirsthead

    \hline
    \textbf{dni\_deudor} & \textbf{tipo} & \textbf{fecha\_grabacion} & \textbf{estado} & \textbf{matricula} \\ \hline
    \endhead
    
    63050842E & discrepancia carne & 2008-03-12 02:04:01 & stand by & 2955AUA \\ \hline
    63050842E & discrepancia carne & 2008-08-05 09:10:31 & stand by & 2955AUA \\ \hline
    63050842E & discrepancia carne & 2008-09-09 14:26:53 & stand by & 2955AUA \\ \hline
    63050842E & discrepancia carne & 2008-09-14 20:23:03 & stand by & 2955AUA \\ \hline
    63050842E & discrepancia carne & 2009-02-13 14:08:57 & stand by & 2955AUA \\ \hline
    63050842E & discrepancia carne & 2009-02-27 20:08:44 & stand by & 2955AUA \\ \hline
    
\end{longtable}
\caption{Resumen de las multas en proceso de pago}
\end{table}

Como se puede observar, todas las filas tienen el estado ``stand by'', que cumple con lo pedido en el enunciado.

\subsection{Querys del análisis estadístico 1}
Se han obtenido los siguientes resultados para las querys del análisis estadístico 1:

\begin{itemize}
    \item \textbf{Multas por marca y modelo}

\begin{table}[H]
\begin{longtable}{l l l}
    \hline
    \textbf{marca} & \textbf{modelo} & \textbf{total\_multas} \\ \hline
    \endfirsthead
    
    \hline
    \textbf{marca} & \textbf{modelo} & \textbf{total\_multas} \\ \hline
    \endhead
    
    Zitron & Tangerine & 1930 \\ \hline
    Bemev & Berlin & 2528 \\ \hline
    Rinaul & Fuente & 3370 \\ \hline
    Zitron & Orange & 1178 \\ \hline
    Escola & Tercius & 841 \\ \hline
    
\end{longtable}
\caption{Resumen de las multas por marca y modelo}
\end{table}

Como se puede observar, se agrupan el número de sanciones por marca y modelo de vehículo de una manera correcta.

    \item \textbf{Multas por color}

\begin{table}[H]
\begin{longtable}{l l}
    \hline
    \textbf{color} & \textbf{total\_multas} \\ \hline
    \endfirsthead
    
    \hline
    \textbf{color} & \textbf{total\_multas} \\ \hline
    \endhead
    
    azul & 1876 \\ \hline
    dorado metalizado & 1352 \\ \hline
    ocre metalizado & 162 \\ \hline
    gris & 3989 \\ \hline
    marron metalizado & 464 \\ \hline

\end{longtable}
\caption{Resumen de las multas por color}
\end{table}

Como se puede observar, se agrupan el número de sanciones por color de vehículo de una manera correcta.

    \item \textbf{Multas de velocidad por marca y modelo}

\begin{table}[H]
\begin{longtable}{l l l}
    \hline
    \textbf{marca} & \textbf{modelo} & \textbf{total\_multas} \\ \hline
    \endfirsthead
    
    \hline
    \textbf{marca} & \textbf{modelo} & \textbf{total\_multas} \\ \hline
    \endhead
    
    Zitron & Tangerine & 1775 \\ \hline
    Bemev & Berlin & 2304 \\ \hline
    Rinaul & Fuente & 3076 \\ \hline
    Zitron & Orange & 1071 \\ \hline
    Escola & Tercius & 753 \\ \hline
\end{longtable}
\caption{Resumen de las multas de velocidad por marca y modelo}
\end{table}

Como se puede observar, se agrupan el número de sanciones de velocidad por marca y modelo de vehículo de una manera correcta. Además, el número de multas es menor al de la tabla de multas por marca y modelo, pudiendo asumir que la selección por el tipo ``velocidad'' ha sido efectiva.

\end{itemize}

\subsection{Querys del análisis estadístico 2}
Se han obtenido los siguientes resultados para las querys del análisis estadístico 2:

\begin{itemize}
    \item \textbf{Exceso de velocidad medio}

\begin{table}[H]
\begin{longtable}{l l l}
    \hline
    \textbf{carretera} & \textbf{media\_velocidad\_registrada} & \textbf{media\_velocidad\_radar} \\ \hline
    \endfirsthead
    
    \hline
    \textbf{carretera} & \textbf{media\_velocidad\_registrada} & \textbf{media\_velocidad\_radar} \\ \hline
    \endhead
    
    A2 & 116 & 100 \\ \hline
    A3 & 116 & 100 \\ \hline
    M30 & 66 & 50 \\ \hline
    M45 & 96 & 80 \\ \hline
    M40 & 96 & 80 \\ \hline
    A6 & 116 & 100 \\ \hline
    A4 & 116 & 100 \\ \hline
    A1 & 117 & 100 \\ \hline
    M50 & 117 & 100 \\ \hline
    A5 & 116 & 100 \\ \hline
    
\end{longtable}
\caption{Resumen de las medias de exceso de velocidad medio}
\end{table}

Estos resultados permiten obtener en la capa de aplicación un porcentaje dividiendo el resultado de ``media velocidad registrada'' entre ``media velocidad radar''. 

    \item \textbf{Tramo y sentido mas conflictivo}

\begin{table}[H]
\begin{longtable}{l l l l}
    \hline
    \textbf{carretera} & \textbf{kilometro} & \textbf{sentido} & \textbf{infracciones\_tramo} \\ \hline
    \endfirsthead
    
    \hline
    \textbf{carretera} & \textbf{kilometro} & \textbf{sentido} & \textbf{infracciones\_tramo} \\ \hline
    \endhead
    
    M45 & 28 & ascending & 202 \\ \hline
    M40 & 43 & descending & 421 \\ \hline
    A1 & 218 & ascending & 184 \\ \hline
    A5 & 53 & ascending & 263 \\ \hline
    M30 & 11 & ascending & 702 \\ \hline
    
\end{longtable}
\caption{Resumen de infracciones por tramo y sentido de carretera}
\end{table}

La query muestra correctamentre el número de infracciones por tramo y dirección de cada carretera, pudiendo escoger el mayor en la capa de aplicación.

\end{itemize}

\subsection{Querys del análisis estadístico 3}
Se han obtenido los siguientes resultados para las querys del análisis estadístico 3:

\begin{itemize}
    \item \textbf{Conductores más infractores}

\begin{table}[H]
\begin{longtable}{l l}
    \hline
    \textbf{dni\_deudor} & \textbf{num\_multas} \\ \hline
    \endfirsthead
    
    \hline
    \textbf{dni\_deudor} & \textbf{num\_multas} \\ \hline
    \endhead
    
    63050842E & 525 \\ \hline
    55074832K & 435 \\ \hline
    78135349W & 472 \\ \hline
    12519376Q & 431 \\ \hline
    28232810B & 837 \\ \hline
    
\end{longtable}
\caption{Resumen de multas por DNI del deudor}
\end{table}

La query funciona correctamente, agrupando el número de sanciones por el dni del deudor.

    \item \textbf{Probabilidad de infracción cuando el conductor es distinto al propietario}

\begin{table}[H]
\begin{longtable}{l l}
    \hline
    \textbf{conductor\_igual\_propietario} & \textbf{count} \\ \hline
    \endfirsthead
    
    \hline
    \textbf{conductor\_igual\_propietario} & \textbf{count} \\ \hline
    \endhead
    
    False & 29656 \\ \hline
    True & 19069 \\ \hline
    
\end{longtable}
\caption{Resumen de las coincidencias entre conductor y propietario}
\end{table}

La queery funciona correctamente, devolvienvdo dos filas con las que calcular un porcentaje en la capa de aplicación.

\end{itemize}

\newpage
\section{Conclusiones}

Las tablas creadas y las consultas diseñadas cumplen con los casos de uso planteados inicialmente, devolviendo los valores esperados y mostrando que el modelo de datos es funcional y alineado con los objetivos establecidos.

\label{sec:conclusion}

\end{document}
